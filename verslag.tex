\documentclass{article}
\usepackage[utf8]{inputenc}

\usepackage{amsmath}

\usepackage{setspace}

\usepackage{listings}
\usepackage{color}
\definecolor{commentc}{rgb}{0, 0.5, 0}
\definecolor{stringc}{rgb}{0.5, 0.1, 0}
\definecolor{keywordc}{rgb}{0, 0.25, 1}
\definecolor{linenumc}{rgb}{0.5, 0.5, 0.5}
\lstset{language=C++, showstringspaces=false, basicstyle={\small \ttfamily},
  numbers=left, numberstyle={\small \color{linenumc} \ttfamily}, numberfirstline=false, breaklines=true,
  stepnumber=1, tabsize=4, 
  commentstyle=\color{commentc},
  stringstyle=\color{stringc}, keywordstyle=\color{keywordc}
}


\renewcommand{\baselinestretch}{1.25} 
\usepackage[
    top    = 1.5cm,
    bottom = 4cm,
    left   = 3.00cm,
    right  = 2.50cm]{geometry}
    
\title{Algoritmiek - Programmeeropdracht Drie}
\author{Jelle van Cappelle \and L\'eon van Velzen}
\date{May 2018}

\usepackage{natbib}
\usepackage{graphicx}

\begin{document}

\maketitle

\section{Introductie}

Een kaartspel wordt als volgt gespeeld. Voor elke soort kaart (schoppen, hart, ruiten, klaveren) liggen er vier rijen op tafel onder elkaar. De kolommen van deze rijen zijn genummerd van 1 tot $n \ge 1$. In elk vakje gevormd door deze rijen en kolommen ligt ofwel een kaart ofwel geen kaart. De speler kan punten verdienen door \emph{rijen} en \emph{sets} te vormen. Een rij bestaat uit drie of meer op\'e\'envolgende kaarten in dezelfde rij. Een set bestaat uit vier dezelfde kaarten in dezelfde kolom. Een kaart in een set of rij levert evenveel punten op als de kolom waarin het zich bevindt. De vraag is nu hoe er sets en rijenuit het spel gehaald moeten worden zodat de score maximaal is. Het bijgevoegde programma berekent deze maximale score en bepaalt welke sets uit het spel gehaald moeten worden om deze maximale score te behalen.

\section{Oplossingsstrategie}

Het algoritme dat door het programma gebruikt wordt berust op het volgende inzicht: een optimale oplossing bevat een set of bevat geen set. Verder moet in het geval dat de oplossing een set bevat de oplossing ook een \emph{meest rechter set} bevatten. We defini\"eren nu de functie maxscore(j) als het maximale aantal punten dat je kunt behalen door sets een rijen te vormen uit de aanwezige speelkaarten met waardes $1, \ldots, j$. Op basis van het vorige inzicht kunnen we deze functie recursief implementeren.
Hiervoor hebben we wel de functie rijscore(i, j) nodig die we defini\"eren als het maximale aantal punten dat je kunt behalen door alleen rijen te vormen uit de aanwezige speelkaarten in kolommen $i + 1, \ldots, j + 1$.  Deze functie kunnen we ook recursief implementeren. Laten we eerst bekijken hoe deze functie ons helpt bij het implementeren van maxscore.

\begin{align*}
\text{maxscore}(0) &= 0 \\
\text{maxscore}(j) &= \text{max}(\text{rijscore}(0, j), \max_{0 \le k \le j \ \land \ \text{mogelijkeset}(k)} \ (\text{maxscore}(j - 1) + 4\cdot (k + 1) +  \text{rijscore}(k + 1, j)))
\end{align*}

Zoals aangegeven stelt $k$ de kolom \emph{index} voor van een mogelijke meest rechter set. In het geval dat $k$ de meest rechter set is kunnen er tussen $k$ en $j$ alleen maar rijen aanwezig zijn. De term $4 \cdot (k + 1)$ stelt de waarde van de set met index $k$ voor. 

De functie rijscore(i, j) kan ook recursief geimplementeerd worden. Laat $k$ het aantal aanwezige kaarten zijn dat zonder onderbrekingen voorafgaat aan de kaart in rij $i$ en kolom $k$. Om de bijdrage van deze kaart aan de score te berekenen moeten we verschillende mogelijkheden voor $k$ apart bekijken.

$$
\text{rijscore\_kaart}(i, j) = \begin{cases}
    0 & \text{als} \ k \le 1 \\
    (j + 1) + j + (j - 1) & \text{als} \  k = 2 \\
    j + 1 & \text{als} \  k \ge 3
\end{cases}
$$

In het eerste geval is het niet mogelijk een rij te vormen. In het tweede geval kan er een rij gevormd worden met de twee voorliggende kaarten. In het laatste geval wordt alleen de score $j + 1$ meegenomen aangezien de rest van de punten voor deze rij al wordt meegenomen in de recursieve aanroep $\text{rijscore}(i, j - 1)$.

$$
\text{rijscore}(i, j) = \text{rijscore}(i, j - 1) + \sum_{i = 0}^{3} \  \text{rijscore\_kaart}(i, j)
$$

\subsection{Het bijhouden van de gebruikte sets}

Zoals hierboven beschreven is $\text{maxscore}(j)$ een recursieve functie die na gaat waar de meest rechter set set gevormd moet worden om de maximale score te behalen met de kaarten met waardes $1, \ldots, j$. Stel dat dit de set van kaarten met waarde $i$ is, dan bepaalt $\text{maxscore}(i - 1)$ waar de op één na meest rechter set moet komen, enzovoorts.
Als voor $\text{maxscore}(i - 1)$ bekend is welke sets gevormd moeten worden om deze score te behalen, dan zijn de sets die gevormd moeten worden om de score van $\text{maxscore}(j)$ te behalen ook te bepalen, namelijk dezelfde sets plus de nieuwe meest rechter set. Wij maken van dit inzicht gebruik in ons programma om een array met gebruikte sets bij te houden voor iedere mogelijke aanroep van $\text{maxscore}(j)$. De array voor $\text{maxscore}(j)$ wordt in dit voorbeeld dus gevuld met een kopie van de array voor $\text{maxscore}(i - 1)$, waarna de nieuwe set (indien die er is) wordt toegevoegd.

\section{Complexiteit}

De complexiteit van ons algoritme is $O(n^3)$. Omdat we de resultaten van iederen aanroep aan $\text{maxscore}(j)$ opslaan in een array (de cache), en omdat $\text{maxscore}(n)$ alleen afhankelijk kan zijn van alle waardes van $\text{maxscore}(j < n)$, is de complexiteit van deze functie dus in het ergste geval gelijk aan de complexiteit van het uitrekenen van iedere $\text{maxscore}(j < n)$ gegeven dat iedere $\text{maxscore}(i < j)$ al bekend is en in de cache zit. De complexiteit $C$ van $\text{maxscore}(n)$ is in dat geval:

$$C_{\text{maxscore}(n)\text{, cache leeg}} = \sum_{j = 0}^{n} \ C_{\text{maxscore}(j)\text{, cache vol}}$$

Wanneer de cache vol zit heeft iedere aanroep van $\text{maxscore}(i < j)$ een complexiteit van $O(1)$, en dus geldt hiervoor:

$$C_{\text{maxscore}(j)\text{, cache vol}} = C_{\text{for-loop}} * (C_{\text{maxscore}(i < j)} + C_{\text{rijscore}}) = O(j) * (O(1) + O(j)) = O(j) * O(j) = O(j^2)$$

De totale complexiteit wordt dus:

$$C_{\text{maxscore}(n)\text{, cache leeg}} = \sum_{j = 0}^{n} \ C_{\text{maxscore}(j)\text{, cache vol}} = O(n) * O(n^2) = O(n^3)$$

\section{Resultaten van ons programma}

\subsection{n = 10}

Maximale score is 146. Hiervoor moet set 10 gebruikt worden.

\lstinputlisting{kaartspel1.in}

\subsection{n = 30}

Maximale score is 1149. Hiervoor moeten er geen sets gebruikt worden.

\lstinputlisting{kaartspel2.in}

\subsection{n = 100}

Maximale score is 4140. Hiervoor moet set 29 gebruikt worden.

\lstinputlisting{kaartspel4.in}

\section{main.cc}

\lstinputlisting{main.cc}

\end{document}
